\appendix{Представление графического материала}

Графический материал, выполненный на отдельных листах,
изображен на рисунках А.1--А.\arabic{числоПлакатов}.
\setcounter{числоПлакатов}{0}

\renewcommand{\thefigure}{А.\arabic{figure}} % шаблон номера для плакатов

\begin{landscape}

\begin{плакат}
    \includegraphics[width=0.82\linewidth]{plakat_title}
    \заголовок{Сведения о ВКРБ}
    \label{pl1:image}      
\end{плакат}

\begin{плакат}
    \includegraphics[width=0.82\linewidth]{plakat_goals}
    \заголовок{Цель и задачи разработки}
    \label{pl2:image}      
\end{плакат}

\begin{плакат}
    \includegraphics[width=0.82\linewidth]{plakat_usecase}
    \заголовок{Диаграмма вариантов использования}
    \label{pl3:image}      
\end{плакат}

\begin{плакат}
    \includegraphics[width=0.82\linewidth]{plakat_concept}
    \заголовок{Концептуальная модель данных}
    \label{pl4:image}      
\end{плакат}

\begin{плакат}
	\includegraphics[width=0.82\linewidth]{plakat_back}
	\заголовок{Диаграмма компонентов серверной части}
	\label{pl5:image}      
\end{плакат}

\begin{плакат}
	\includegraphics[width=0.82\linewidth]{plakat_map}
	\заголовок{Карта маршрутов}
	\label{pl6:image}      
\end{плакат}

\begin{плакат}
	\includegraphics[width=0.82\linewidth]{plakat_class}
	\заголовок{Диаграмма классов предметной области}
	\label{pl7:image}      
\end{плакат}

\begin{плакат}
	\includegraphics[width=0.82\linewidth]{plakat_end}
\заголовок{Заключение}
\label{pl8:image}      
\end{плакат}


\end{landscape}
