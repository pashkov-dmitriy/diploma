\section{Анализ предметной области}

\subsection{Понятия и принципы работы музыкальных сервисов}

Музыкальные сервисы — это цифровые платформы, предназначенные для распространения, доступа и управления музыкальным контентом. Они позволяют пользователям слушать музыку в режиме онлайн через потоковую передачу (стриминг) или скачивать треки для офлайн-прослушивания. Эти сервисы стали важной частью современной музыкальной индустрии, обеспечивая легкий доступ к огромному музыкальному каталогу\cite{mus1}.

Центральным компонентом музыкальных сервисов является предоставление аудио конечному пользователю. Это может быть реализовано как загрузка аудиофайла на устройство клиента или как потоковая передача, позволяющая пользователю начинать прослушивание аудио, не дожидаясь конца загрузки. Современные музыкальные сервисы обычно реализуют оба подхода. Также важной частью приложений данной категории является навигация и поиск в музыкальном каталоге. Все треки имеют метаданные для обеспечения структуризации предоставляемых данных. Зачастую музыкальные сервисы реализуют функции, помогающие пользователям самостоятельно реализовывать структуру треков в своей медиатеке. Примером такой функции являются "лайки" - отметки, которые пользователи могут ставить на треки. Затем отмеченные треки могут быть получены пользователем. Также распространена структуризация с помощью плейлистов. Плейлист — это упорядоченный список музыкальных треков, составленный на основе определённых критериев или предпочтений. Плейлисты могут быть созданы пользователями для личного использования, кураторами для общественного вещания или автоматически сгенерированы музыкальными сервисами на основе алгоритмов анализа предпочтений слушателей. Поиск в каталоге осуществляется по названиям треков, исполнителей, альбомов и плейлистов. Музыкальные сервисы могут содержать социальные функции, например комментарии для альбомов или возможность поделиться плейлистом с другим пользователем системы или в социальных сетях. 

Сравнение музыкальных веб-приложений с традиционными способами доступа к музыке, такими как покупка компакт-дисков, выявляет ряд преимуществ и недостатков. Одним из ключевых преимуществ является удобство: пользователи могут получать доступ к миллионам треков из любой точки мира, где есть доступ в интернет. Это также экономически выгодно, так как пользователи платят за подписку, а не за каждый трек или альбом. Однако это влечёт за собой потенциальные недостатки, такие как зависимость от стабильного интернет-соединения и возможные ограничения на доступ к определённым трекам или альбомам в зависимости от географического местоположения или условий лицензионных соглашений.

\subsection{История развития цифрового аудио и музыкальных сервисов}

История развития музыкальных сервисов охватывает переход от физических носителей к цифровой дистрибуции музыки, что существенно изменило взаимодействие пользователей с музыкальным контентом. Изначально музыка распространялась через такие физические носители, как виниловые пластинки, кассеты и компакт-диски, которые требовали физического владения и использования специального оборудования для воспроизведения.

С развитием интернет-технологий и повышением скорости интернет-соединений в конце 1990-х — начале 2000-х годов начался переход к цифровому распространению музыки. Первые музыкальные сервисы, такие как Napster, предлагали P2P-платформы для обмена файлами, что вызвало значительные споры в области авторских прав. Это привело к законодательным изменениям и разработке новых моделей лицензирования музыкального контента.

Прорывом в легализации цифровой музыки стал запуск iTunes Store компанией Apple в 2003 году, который предложил модель покупки отдельных музыкальных треков и альбомов. Это послужило основой для развития других музыкальных платформ и укрепления цифровой дистрибуции как доминирующего метода в музыкальной индустрии.

Следующим значительным этапом стало появление сервисов стриминга, таких как Spotify и Pandora, в середине 2000-х. Эти сервисы предложили модель подписки, позволяющую пользователям получать неограниченный доступ к музыкальным библиотекам без необходимости покупки каждого трека. Такая модель предоставила пользователю возможность экспериментировать с разнообразным музыкальным контентом и способствовала росту индивидуализации музыкального опыта через алгоритмы рекомендаций.

Появление музыкальных сервисов с подпиской также привело к изменениям в структуре доходов музыкальной индустрии, где значительная часть доходов стала приходиться на стриминг, а не на продажи физических носителей или цифровых копий. Современные технологии также позволили интегрировать музыкальные сервисы с различными устройствами и платформами, расширяя доступность и функциональность для конечного пользователя.

Таким образом, развитие музыкальных сервисов отражает более широкую тенденцию цифровизации в медиа и культуре, где удобство использования, доступность и персонализация стали ключевыми факторами в определении форматов потребления музыкального контента.

\subsection{Анализ рынка музыкальных сервисов}

В 2015 году стриминговые сервисы впервые стали основным источником дохода для музыкальных лейблов и исполнителей. Согласно данным RIAA, в этом году стриминги составили 34,3\% от общего объема выручки музыкальной индустрии, тогда как доходы от сервисов загрузки музыки, таких как iTunes, достигли 34\%. Продажи музыки на физических носителях принесли 28,8\% дохода.
В последнее время отмечается тенденция к уменьшению рынка компакт-дисков и падению продаж цифровой музыки, в то время как доходы от потоковой трансляции музыки продолжают расти. К 2017 году также значительно увеличилась популярность сервисов с платной подпиской. В 2016 году Россия заняла пятое место по числу подписчиков на «Apple Music» после США, Великобритании, Японии и Канады. По данным IFPI за 2017 год, на первом месте по количеству подписчиков в России находится «Apple Music» с 600 тыс. пользователей, на втором — Яндекс.Музыка с 250 тыс., а на третьем — Google Play Music с 100 тыс. подписчиков (согласно рисунку 3). Эксперты оценивают годовой доход этих сервисов от подписок примерно в 2,4 млрд руб. (34 млн долл.)\cite{mus2}.

Согласно данным от IFPI, организации, представляющей глобальную индустрию звукозаписи, мировой рынок музыки вырос на 7,4\% в 2020 году, увеличиваясь уже шестой год подряд. Глобальный музыкальный отчёт IFPI, опубликованный сегодня, указывает, что общая выручка за 2020 год достигла 21,6 миллиарда долларов США.
Рост был обусловлен музыкальным стримингом, причём особенно значительно увеличились доходы от платных подписок, которые выросли на 18,5\%. К концу 2020 года число платных подписок составило 443 миллиона аккаунтов. Всего доходы от стриминговых сервисов, включая как платные подписки, так и сервисы с рекламной поддержкой, увеличились на 19,9\%, составив 13,4 миллиарда долларов или 62,1\% от всего мирового дохода от звукозаписи. Прирост доходов от стриминга более чем компенсировал падение в других сегментах, включая 4,7\%-ное снижение физических продаж и 10,1\%-ное уменьшение доходов от прав на исполнение, большей частью вызванное пандемией COVID-19\cite{ifpi}.

\subsection{Обзор технологий, используемых в музыкальных сервисах}

В основе музыкальных сервисов лежит клиент-серверная модель, где сервер обрабатывает запросы от клиентов, выполняя функции хранения, поиска и передачи музыкальных треков. 
Серверная часть отвечает за основной функционал приложения. Особенностью музыкальных сервисов является необходимость работы с большим количеством медиафайлов, их количество может составлять десятки миллионов. Для этого требуется использование специальных технологий, таких как распределенные файловые системы, объектные хранилища, сети доставки контента. Серверная сторона часто развертывается на облачных платформах, таких как AWS (Amazon Web Services), Google Cloud или Microsoft Azure, что позволяет масштабировать ресурсы в соответствии с изменяющейся нагрузкой и обеспечивать высокую доступность сервиса.
Конечный пользователь взаимодействует с клиенской частью приложения. В качестве клиентской части используются веб- и мобильные приложения.
Веб-приложения реализуются с использованием таких технологий, как HTML5, CSS3 и JavaScript. В основном используются фронтенд-фреймворки,  такие как React.js, Vue.js или Angular, обеспечивающие создание динамических и адаптивных пользовательских интерфейсов, кроссплатформенную совместимость и оптимизированный пользовательский опыт на различных устройствах. 

Для передачи музыкального контента используются различные протоколы стриминга, такие как HLS (HTTP Live Streaming) и DASH (Dynamic Adaptive Streaming over HTTP). Эти протоколы позволяют эффективно передавать аудио- и видеоконтент через интернет, адаптируя качество потока в зависимости от скорости соединения и возможностей устройства пользователя.

Базы данных играют ключевую роль в управлении музыкальными данными, такими как информация о треках, пользователях, плейлистах и предпочтениях. NoSQL базы данных, такие как MongoDB или Cassandra, часто используются для управления большими объемами структурированных и неструктурированных данных, что позволяет обеспечить высокую производительность и масштабируемость.

Алгоритмы машинного обучения и искусственного интеллекта применяются для анализа пользовательских данных и формирования персонализированных рекомендаций. Системы рекомендаций анализируют предпочтения и поведение пользователей, предлагая им музыкальный контент, который может их заинтересовать.