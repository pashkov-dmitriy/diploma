\abstract{РЕФЕРАТ}

Объем работы равен \formbytotal{lastpage}{страниц}{е}{ам}{ам}. Работа содержит \formbytotal{figurecnt}{иллюстраци}{ю}{и}{й}, \formbytotal{tablecnt}{таблиц}{у}{ы}{}, \arabic{bibcount} библиографических источников и \formbytotal{числоПлакатов}{лист}{}{а}{ов} графического материала. Количество приложений – 2. Графический материал представлен в приложении А. Фрагменты исходного кода представлены в приложении Б.

Перечень ключевых слов: веб-сервис, музыкальное воспроизведение, разработка, программное обеспечение, веб-приложение, пользовательский интерфейс, аудио поток, база данных, музыкальные жанры, авторизация, аутентификация, стриминговая технология, API, диаграмма.

Объектом разработки является веб-сервис для прослушивания музыки.

Целью выпускной квалификационной работы является является разработка и реализация веб-сервиса для прослушивания музыки, который предоставит пользователям  доступ к музыкальному контенту, обеспечит высокое качество звука и предоставит функционал для создания и управления персональными плейлистами.

При разработке приожения были выделены основные сущности предметной области, разработаны классы и методы модулей, обеспечивающие работу с сущностями предметной области, а также корректную работу web-сайта, разработаны страницы, содержащие информацию о исполнителях, их альбомах, плейлистах и пользователях, разработны модули для организации потоковой передачи аудио.

\selectlanguage{english}
\abstract{ABSTRACT}
  

The volume of work is \formbytotal{lastpage}{pages}{}{}. The work contains \formbytotal{figurecnt}{illustrations}{}{}, \formbytotal{tablecnt}{tables}{}{}, \arabic{bibcount} bibliographic sources, and \formbytotal{числоПлакатов}{sheets}{}{}. The number of appendices is 2. The graphical material is presented in Appendix A. Fragments of source code are presented in Appendix B.

List of keywords: web service, music playback, development, software, web application, user interface, audio stream, database, music genres, authentication, authorization, streaming technology, API, diagram.

The object of development is a web service for music playback.

The aim of the graduation project is the development and implementation of a web service for music playback, which will provide users access to music content, ensure high-quality sound, and provide functionality for creating and managing personal playlists.

During the development of the application, the main entities of the subject area were identified, classes and methods of modules were developed to work with the entities of the subject area, as well as to ensure the correct operation of the web site. Pages containing information about artists, their albums, playlists, and users were developed, and modules were developed to organize the streaming of audio.
\selectlanguage{russian}
