\section*{ВВЕДЕНИЕ}
\addcontentsline{toc}{section}{ВВЕДЕНИЕ}

В современном мире веб-сервисы для прослушивания музыки занимают важное место в повседневной жизни людей. С развитием интернета и цифровых технологий потребление музыкального контента значительно изменилось. Пользователи все чаще предпочитают онлайн-сервисы, предоставляющие доступ к обширным библиотекам музыкальных произведений, вместо традиционных методов прослушивания, таких как физические носители или скачивание файлов. Это обусловлено удобством, разнообразием выбора и возможностью прослушивания музыки в любой момент времени и в любом месте.
На момент последних моих данных музыкальные сервисы продолжают развиваться и конкурировать за пользователей с помощью различных инноваций и функций. 
Ведущие платформы, такие как Spotify, Apple Music, YouTube Music и Amazon Music, продолжают оставаться в лидирующих позициях, предлагая широкий выбор музыкального контента и персонализированные рекомендации.
Новые игроки и региональные сервисы также продолжают появляться, усиливая конкуренцию на рынке и способствуя развитию инноваций в музыкальной индустрии. 
Перспективы музыкальных сервисов остаются выскими, поскольку они продолжают приспосабливаться к изменяющимся потребностям пользователей и технологическим инновациям. Развитие их функциональности, контента и доступности на различных устройствах будет ключевым направлением. Инновации в области рекомендательных систем и персонализации контента также сделают сервисы более привлекательными для аудитории. Кроме того, углубление интеграции с другими платформами и сервисами могут стимулировать рост и конкурентоспособность музыкальных сервисов.

\emph{Целью данной работы} является разработка и реализация веб-сервиса для прослушивания музыки, который предоставит пользователям  доступ к музыкальному контенту, обеспечит высокое качество звука и предоставит функционал для создания и управления персональными плейлистами. Для достижения поставленной цели необходимо решить \emph{следующие задачи:}
\begin{itemize}
	\item провести анализ предметной области; определить ключевые особенности предметной области и перспективы программного проекта;
	\item разработать модель данных прграммной системы; определить ключевые сущности; разработать проект базы данных;
	\item спроектировать клиентскую и серверную части программной системы; на основе требований пользователей спроектировать пользовательский интерфейс;
	\item произведсти системное тестирование системы; написать модульные тесты.
\end{itemize}

\emph{Целью данной работы} является разработка и реализация веб-сервиса для прослушивания музыки, который предоставит пользователям доступ к музыкальному контенту, обеспечит высокое качество звука и предоставит функционал для создания и управления персональными плейлистами. Для достижения поставленной цели необходимо решить \emph{следующие задачи:}
\begin{itemize}
	\item провести анализ предметной области; определить ключевые особенности предметной области и перспективы программного проекта;
	\item разработать модель данных программной системы; определить ключевые сущности; разработать проект базы данных;
	\item спроектировать клиентскую и серверную части программной системы; на основе требований пользователей спроектировать пользовательский интерфейс;
	\item произвести системное тестирование системы; написать модульные тесты.
\end{itemize}

\emph{Структура и объем работы.} Отчет включает введение, четыре раздела основной части, заключение, список использованных источников и два приложения. Текст выпускной квалификационной работы занимает \formbytotal{lastpage}{страниц}{у}{ы}{ы}.

\emph{Во введении} определена цель работы, поставлены задачи разработки, описана структура работы и дано краткое содержание каждого раздела.

\emph{В первом разделе} проводится исследование предметной области разрабатываемой системы.

\emph{Во втором разделе} производится формулировка требований, моделирование вариантов использования.

\emph{В третьем разделе} описываются выбранные технические решения и проект данных разрабатываемой системы.

\emph{В четвертом разделе} описываются классы и их методы, использованные при разработке, а также проводится тестирование созданного приложения.

В заключении подведены итоги проделанной работы, полученные в ходе разработки.

В приложении А представлен графический материал.
В приложении Б содержатся фрагменты исходного кода.
