\addcontentsline{toc}{section}{СПИСОК ИСПОЛЬЗОВАННЫХ ИСТОЧНИКОВ}

\begin{thebibliography}{9}
    \bibitem{mus1} А.А. Ивлева МУЗЫКАЛЬНЫЕ СТРИМИНГОВЫЕ СЕРВИСЫ КАК НОВЫЙ ИНСТРУМЕНТ ПРОДВИЖЕНИЯ БРЕНДОВ // Экономика и бизнес: теория и практика. 2021. №9-1. URL: https://cyberleninka.ru/article/n/muzykalnye-strimingovye-servisy-kak-novyy-instrument-prodvizheniya-brendov (дата обращения: 14.05.2024).
	\bibitem{mus2} Миннебаева Р.А., Миннебаев Р.А. АНАЛИЗ РЫНКА СТРИМИНГОВЫХ СЕРВИСОВ // Форум молодых ученых. 2018. №5-2 (21). URL: https://cyberleninka.ru/article/n/analiz-rynka-strimingovyh-servisov (дата обращения: 14.05.2024).
	\bibitem{ifpi} IFPI issues Global Music Report 2021 // IFPI URL: https://www.ifpi.org/ifpi-issues-annual-global-music-report-2021/ (дата обращения: 12.05.2024).
    \bibitem{springboot}  Уоллс, К. Spring в действии / Крейг Уоллс. - 6-е изд. – Москва : ДМКПресс, 2022. – 544 с. - ISBN 978-5-93700-112-2. - Текст : непосредственный.
    \bibitem{springcloud} Карнелл, Дж. Микросервисы Spring в действии / Джон Карнелл, Иллари Уайлупо Санчес. – Москва : ДМК Пресс, 2022. – 490 с. - ISBN 978-5-97060-971-2. - Текст : непосредственный.
    \bibitem{java} Шилдт, Г. Java полное руководство / Герберт Шилдт. – Москва : ООО "И.Д. Вильяме", 2015. – 1376 с. - ISBN 978-5-84-591759-1. - Текст : непосредственный.
    \bibitem{jshtml} Коэн, И. Полный справочник по HTML, CSS и JavaScript / Лазаро Исси Коэн, Джозеф Исси Коэн. – Москва : Эксмо, 2017. – 246 с. - ISBN 978-5-9790-0009-1. - Текст : непосредственный.
    \bibitem{js} Хорстманн, К. Современный JavaScript для нетерпеливых / Кей Хорстманн. — Москва : ДМК Пресс, 2021. — 288 с. — ISBN 978-5-97060-177-8. — Текст : электронный // Лань : электронно-библиотечная система. — URL: https://e.lanbook.com/book/190715 (дата обращения: 15.05.2024). — Режим доступа: для авториз. пользователей.
    \bibitem{vue} Vue.js. Guide: Introduction to Vue.js [Электронный ресурс]. URL: https://vuejs.org/guide/introduction.html (дата обращения: 26.04.2024).
    \bibitem{ui} Мандел, Т. Разработка пользовательского интерфейса / Т. Мандел. – Москва : ДМК Пресс, 2019. – 420 с. – ISBN 978-5-04-195060-6. – Текст : непосредственный.
    \bibitem{ui2} Купер, А. Интерфейс. Основы проектирования взаимодействия / А. Купер, Р. Рейман, Д. Кронин, К. Носсел – 4-е изд. – Санкт-Петербург : Питер, 2021. – 720 с. – ISBN 978-5-4461-0877-0. - Текст : непосредственный.
    \bibitem{arch1} Назаров, С.В. Архитектура и проектирование программных систем: монография / С.В. Назаров. – Москва : Инфра-М, 2016. – 374 с. - ISBN 978-5-16-011753-9. – Текст : непосредственный.
    \bibitem{arch2} Кугушева, Д.С. Проектирование сложного программного обеспечения с использованием микросервисной архитектуры // Инновации и инвестиции. 2020. №5. URL: https://cyberleninka.ru/article/n/proektirovanie-slozhnogo-programmnogo-obespecheniya-s-ispolzovaniem-mikroservisnoy-arhitektury (дата обращения: 27.04.2024).
    \bibitem{persist} Бауэр, К. Java Persistence API и Hibernate/ К. Бауэр, Г. Кинг. – Москва : ДМК Пресс, 2018. – 632 с. – ISBN 978-5-97060-674-2. – Текст : непосредственный.
    \bibitem{sql1} Грофф, Д.Р. SQL. Полное руководство / Д. Р. Грофф, П.Н. Вайнберг, Э.Д. Опель. – 3-е изд.. – Санкт-Петербург : Диалектика, 2019. – 560 с. - ISBN 978-5-90-711426-5 - Текст : непосредственный
    \bibitem{nosql} Григорьев, Ю.А. Реляционные базы данных и системы NoSQL: учебное пособие / Ю.А. Григорьев, А.Д. Плутенко, О.Ю. Плужникова. – Благовещенск : Амурский гос. ун-т, 2018. – 424 с. - ISBN 978-5-93493-308-2. - Текст : непосредственный.
    \bibitem{spoti} Web API | Spotify for Developers // Spotify for Developers URL: https://developer.spotify.com/documentation/web-api (дата обращения: 30.04.2024).
    \bibitem{uml} Буч, Г. Введение в UML от создателей языка / Г. Буч, И. Якобсон, Д. Рамбо. – Москва : ДМК Пресс, 2015. – 498 с. – ISBN 978-5-457-43379-3. – Текст : непосредственный.
    \bibitem{uml2} Завьялов, А. В. Диаграммы UML для анализа и проектирования информационных систем : учебно-методическое пособие / А. В. Завьялов. — Москва : РТУ МИРЭА, 2021. — 65 с. — Текст : электронный // Лань : электронно-библиотечная система. — URL: https://e.lanbook.com/book/218630 (дата обращения: 05.05.2024). — Режим доступа: для авториз. пользователей.
    \bibitem{webapi} Лоре, А. Проектирование веб-API / А. Лоре. – Москва : ДМК Пресс, 2020. – 440 с. – ISBN ISBN 978-5-97060-861-6. – Текст : непосредственный.
    \bibitem{http} Поллард, Б. HTTP/2 в действии/ Б. Поллард. – Москва : ДМК Пресс, 2021. – 424 с. – ISBN 978-5-97060-925-5. – Текст: непосредственный.
    
\end{thebibliography}
